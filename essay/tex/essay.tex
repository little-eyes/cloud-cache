\section{Introduction}
In recent years, cloud computing has provided us with a lot of benefits in many area. Major cloud computing vendors like Amazon, Google and Dropbox have offered the users around the world a cheap, convenient and secure computing and storage resource. For example, Amazon EC2\cite{garfinkel2007evaluation} is one of the most popular cloud computing platform for many companies and individuals. The various types of virtual instances at different price strategies provide very flexible choices for the users. I rent a \emph{m1.micro} instance for the use of my person website and blog for three years at a decent price around $\$100$. With the virtual instance, I do not need to install a physical machine at my home and maintain it day by day, which definitely costs more than $\$100$ for three years. Besides individual users like me, many companies either big ones like Netflix or startups like Pinterest choose Amazon AWS as their dedicated IT infrastructure to reduce the cost of maintaining a complete IT infrastructure by themselves. 

Another scenario of cloud computing's application could be cloud storage. Either Dropbox or Google Drive or Amazon Cloud Storage, they provide both companies and individuals cheap and convenient cloud storage which we do not need to worry about the data loss or accessibility. Even though three of them have different data privacy terms, people still quite enjoy this type of services. For example, Dropbox has a marketing plan for university students who has a \emph{.edu} email will obtain an extra $3$GB plus his own $2$GB storage space for free. When I look around with other students, most of them seems to be very happy with the current services. They use Dropbox as a USB drive to share or transfer large amount of data over the air; they also use Dropbox to show pictures or documents during their presentation. Thus, it seeems that the free $5$GB \emph{USB-air-drive} (which I called for Dropbox) could become more popular and convenient than a $\$10$-$\$15$ USB flash drive.

Event if cloud computing had achieved so many advantages all over these years, we may notice that the current stage of cloud computing has influenced very limited industries, most of them are concentrated in Internet or Web related businesses. However, I think the cloud computing could power more industries in the next $10$ year.   

\section{Mobile and Cloud Computing}
The reason I first discuss the mobile technology and cloud computing is simply because I have observed the big trend of the mobile commercial electronics. Smartphones, tablets and ultra-books are becoming more and more popular in recent several years. Android and iOS powered devices attracts much more attentions than a PC or laptop. These devices are highly mobile, equipped with high quality camera and GPS which makes the device a central hub for the multiple other equipments like camera or GPS navigators. Additionally, the Android and iOS app eco-system have made the smartphones much more valuable than a simple phone. We can see people playing Angry Birds when they are taking a bus and listening to the music when they are exercising. The whole category of educational apps confirm me the significance of smart devices on education. Therefore, it is reasonable to see the mobile devices could somehow dominate the market in the future.

Interestingly, if we look at the app stores, you can always see the shadows of cloud computing. I say ``shadow'' because there is not too many devices directly claim they belongs to cloud computing. But many of them use cloud computing heavily as their backend support. For example, the new private social network app Path\cite{path} provides excellent social network experience via smartphones with close friends and families. Their backend systems such as data storage, machine learning/data mining and message exchanges undoubtedly depend on Amazon AWS infrastructure. Another example is the intelligent social media reader Prismatic\cite{prismatic} on smarthpones. This reader learns the user's interests and topics among numerous posts and streams online, then it starts filtering the news and presents the most relevant news to the users. Prismatic's app on the smartphone is very simple, but the backend systems also rely on Amazon AWS because it has to deal with large scale machine learning and realtime data service. These two components require super powerful computation, reliable data storage and scalable front-end service, which is hard for a startup to build from the scratch. 

Furthermore, some challenges in the mobile computing are still unresolved. For example, the mobile network latency and battery issues which are critical to the usage fluency of mobile applications. To cope with the issues, some researchers propose the concept of \emph{Cloudlets}~\cite{satyanarayanan2009case} which bridges the mobile devices and the clouds through a middle station. To deal with the battery issues, some researchers proposes the \emph{offloading}~\cite{kumar2010cloud} techniques to transfer the computation from the mobile devices to the cloud. To speed up the mobile web browsing experience, Chrome patches the cloud speeding techniques by prefetching web pages in Google's cloud to speed up the webpage loading speed on the mobile devices. Therefore, we could see the cloud computing is not only a backend supporter but a key to solve problems in mobile computing as well.

\section{Education and Cloud Computing}
At the current stage, cloud computing seems to have little interactions with education field, especially the primary and secondary educations. Perhaps the email system could be the only possible connections for the public schools and cloud computing, although not all the public school system will use a cloud email services from Google or Microsoft. My wife is a substitute teacher in Knox county school and she was a substitute teacher in Webb School at Knoxville as well. She tells me there is a big difference of the education philosophy and approach between Knox county school and Webb school. For Webb school at Knoxville, teaching has been computerized. The teachers prepare their course material with iPad or laptop, and every student uses an iPad in class for reading, interactions and homeworks. There is no physical textbooks really necessary in my wife's teaching at Webb school.

In addition to the course material, Webb school has a strong IT team to support the teachers and students. At the new teacher training, the IT stuff at Webb school provides numerous apps inside the Apple store which could potentially benefits their teaching. For example, the students usually finish their homework on iPad and submit the assignment through an app, the teacher can grade it and analyze the performance of the student from its history. One thing I am impressed by this school is that the IT team did not create their own infrastructure in school. However, they use cloud computing as their complete IT infrastructure. For instance, the email system is provided by Google; the reading assignment system is provided by GoodReader, an impressive reading experience provider on mobile devices; student uses eBagpack as an educational Dropbox to management their study materials; and they use edmodo as a educational social network for student to vote and for teacher to post updates about the course. 

At present, very few schools like Webb school have the opportunity to use cloud computing to support both the school daily management and teachings/learnings. Let us take the Knox county school as an example. The school management system  and other IT infrastructure is maintained by the school's IT team. Perhaps it is a reliable system, but it definitely cost more than using a cloud vendor's product. On the other hand, the students in the public school still use textbook to learn and the teachers handle the teachings, assignments and homework traditionally. One of my friends, he teaches fourth grade at Knox county school. He is not satisfied with the current system he has in class, even though he already uses Smartboard (a computerized white board) to teach. He hopes to have a better tool to keep track of the student's performance throughout these years instead of using Excel. He thinks the cost may be one of the major reason for the school to decide build a new system.

Note that the difference between independent schools and public school is not to show how well the independent school is, but to illustrate the trend of cloud computing and education which the cloud computing can help improve the education strategy in the next $10$ years.

On one hand, more and more schools realize the importance of IT infrastructure in class. We need to teach the students how to organize the information well at the beginning. An easy way to start with is to teach them how to use electronic materials and make them familiar with this learning style. From this point of view, the schools need a new system to adapt the changes and needs, but the cost is non-trivial if every school builds their own. 

On the other hand, fortunately, many startups have started to build such a system under the schools' needs. Each individual product can be used by different schools and students that all of them actually share the same system. The products like eBackpack or edmodo are almost free to everyone to use or small amount of fee is charged, which definitely can reduce the cost for the schools to replace the old system. Besides, the cloud computing vendors can provide those startups many advantages, such as agile development and deployment, low computation and storage cost and almost zero IT maintenance cost, so that the product vendors can run their service or product at their lowest cost. For example, Amazon AWS and Google App Engine are the most popular service providing environment for many startups.

Therefore, it is possible to expect that many companies, especially startups, could join the business to provide cheap and convenient education service in cloud to the majority of schools.

In addition to the primary and secondary education system, the cloud computing also benefits the higher education system. Universities already have some basic infrastructure deployed in clouds. For example, the \emph{volmail} system in UT will be upgraded into Microsoft's Office365 platform which is Microsoft's in-cloud office software system. However, the cloud computing's impact in universities are not limited in IT infrastructure.

Massive open online courseware (MOOC) has become a hot area in higher education recently. Typical companies like Coursera provides open and free online classes to students around the world. According to an answer from a Coursera's engineer on Quora\cite{quora2coursera}, their site relies heavily on Amazon AWS platform because each online course usually contains many videos and need a scalable student interaction platform. What coursera can bring to a student now is the accessibility to many different courses from various universities even if the students' universities do not open such a course. For example, I am taking Natural Language Processing and Probabilistic Graphic Model courses on Coursera because we do not have such courses in UT. But, in my opinion, MOOC is changing the way students learn knowledge. 

First, the students can arrange the lecture time by himself within a week so that the time does not conflict with their courses at school. Second, many more students register the class at once, then the analysis will give more accurate information on how well the student performs. Third, the professors can experiment teachings in a more flexible ways, for instance, the students can review other's assignments or projects instead of the professor himself. Fourth, the students can access the course materials anywhere instead of being limited in the classroom.

However, my viewpoint does not necessarily mean the MOOC will replace university some day. In contrast, I would say never. One important phase in learning is the interactions between students and professors. Even if MOOC can provide the platform for such an interaction (not a big problem in cloud computing), the reality is that one or two professors cannot handle thousands of students at a time. Therefore, MOOC could be a good learning makeups in higher education to enjoy different teaching styles and viewpoints. Besides, the open source MOOC platform Class2Go\cite{class2go} makes it much easier for anyone who wants to make an online course for others.

\section{Local Small Business and Cloud Computing}
How the business will use cloud computing in the future $10$ years is a board topic, but I am interested in how it will impact the local small businesses. 

%online-to-offline (O2O) business. The O2O business is the field that business connect online demands and offline convenience together, such as you can accurately discover a local product through online search or recommendation engines (sounds like an advanced e-business). 

In fact, most local business either have their own website to present their products to consumers online or they have use the social media to advertise itself. However, they do not have a complete IT infrastructure stack for their business, especially the small business. Medium and big companies have their IT department or some vendors will provide the service for them. Therefore, those companies have their website, inventory management system, and business intelligence component. In addition, they can use eBay or Amazon to do better marketing work. In contrast, if we look at the local small businesses, such as a small local antique store, they usually do not have such a complete IT infrastructure. Cragilist is the most common used platform for them to advertise their product; and they may buy some basic inventory management system. As for the business intelligence, it is almost impossible for them to take advantage of this valuable tool.

The major reason for the small business to build a IT infrastructure is probably the cost. We cannot expect a local small store to spend thousands of dollars on the IT infrastructure, if they cannot make ten thousands of dollars profit per year. Fortunately, cloud computing could help them have a cheap and efficient IT infrastructure which could potentially improve their business over the years. 

First, in my opinion, an intelligent marketplace is critical for the local business to broadcast their products. Some pioneer companies have entered into this field from different angels. Yelp is one of the local life information search engine which provides people with local restaurant, entertainment and special offers for the smartphone users. The cost for listing the products is free for Yelp and extra fees are charged for premium features. Zarrly is another company trying to build a bi-directional marketplace so that both buyers and sellers can post information and they provide a match engine to bring potential buyers and sellers together intelligently. These companies take advantages of cloud computing to establish their marketplace so that the local or small businesses can spend very little money to buy their services. 

Second, an inventory management system and payment system are also very important to help the growth of a business. The typical cost of such a system is from several hundreds of dollars to thousands of dollars. Although the inventory management system may not be a big obstacle, the small or local business may expect an even cheaper solution. As for the payment system, not all local businesses support all the credit cards. For example, a very good Asian grocery store in Knoxville does not accept the American Express credit cards which make me very uncomfortable to buy something from their store. With the growing of mobile payment, companies like Google and Square has setup a convenient way to pay from smartphones or through the cloud. If the small stores in Knoxville can use their services, we do not need to worry about bringing cash or Visa cards before shopping in those stores.

Finally, the business intelligence system seems to be impossible for the small business right for two reasons: 1) the cost usually is usually not trivial; 2) the small businesss needs to use the inventory management and payment systems to keep track of their data in store but they usually do not have a complete service ready. However, if cloud computing can help the inventory management and payment systems, the business intelligence component is not a big deal, because it is just another machine learning or data mining component plug into the existing system in cloud.

However, the existing companies do not cover all those aspects. For example, eBay or Amazon could cover the intelligent marketplace, inventory management and payment system, but the business intelligence has not been under consideration yet. Google's Big Query is specifically for the analysis part but Google does not have a marketplace or inventory management system provided for the business. Even though I did not find any startups are working on the all-in-one in-cloud full stack infrastructure, but such a infrastructure from online marketplace, to inventory management, to the payment system and business intelligence could benefits the small business not only a cheap IT infrastructure but also make their whole business simpler and easier. 

\section{Conclusion}
In this essay, I discussed some potential impacts that cloud computing cloud benefits for the academia and business in the next $10$ years. For the academia, what the primary and elementary education could benefit from cloud computing is the school IT infrastructure and teaching strategy. Paper-less course material and in-cloud course management system cloud replace the current ones in public school; and what the higher education could take advantages is the massive open online courseware which could bring more diversity to university students and make up unavailable courses in universities. On the other hand, for the business, I think the local small business could enjoy a lot from the cloud computing which could provide them an all-in-one in-cloud IT infrastructure at very low cost so that the local business could connect to both the online customers and offline ones very well.
