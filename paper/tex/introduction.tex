\section{Introduction}\label{sec:introduction}
Cloud computing has become one of the most powerful tools which could potentially help people solve some extremely hard problems. The computation and storage resource in the cloud are relatively cheap and convenient comparing to build something from the scratch. Besides, the public cloud vendors, like Amazon and Google, have made its own powerful computation and storage infrastructure available to the public via cloud computing service. Those cloud vendors provide either virtual machines or pre-defined cloud APIs to help people build application on top of their cloud very easy and fast.

The hard problems (usually belongs to the NP-Complete category like 3-SAT) often require exponential computation time to traverse the solution space in a brute-force manner. In addition, people may need to solve the same types of problems again and again in different instances. Some instances or subproblems possibly have been solved by others. Therefore, we come to the idea that \emph{cache the existing solutions to those problems and query the solutions before we solve any other problems}. 

One immediate challenge of this idea is the storage requirement. Even the same type of problem, e.g. 3-SAT, could have numerous problem instance, which may require several disks to store them, let alone we have many different types of hard problems. Cloud computing, at this point, could be a proper paradigm because it is much less expensive to build a large scale storage and its management system. The other challenge is how we abstract the hard problems in a way that the platform we build could be more general to many different types of hard problems.

In this document, we propose \emph{CloudCache} framework which is built on top of general cloud computing platform and provides the interface to query, cache and solve hard problems in an easier way. The CloudCache framework abstracts the \emph{problem-solution} pairs into \emph{key-value} objects, and uses some object storage system to cache existing problems and solutions. Further, the CloudCache framework provides a general problem solving interface \emph{kernel-solver}, so that the developer can easily write a problem's solution and solve it within the CloudCache's framework. CloudCache executes kernel-solver in a distributed way which tries to maximize the framework's problem solving throughput.

The rest of the document is arranged in the following way: Section~\ref{sec:design} describes the design and reasoning of the CloudCache framework; Section~\ref{sec:implementation} shows the implementations of the framework; Section~\ref{sec:evaluation} deploys the framework in Amazon EC2 and solves $1,000,000$ 3-SAT problem instances; Finally, we discuss some potential improvements and conclusion in Section~\ref{sec:conclusion}.
